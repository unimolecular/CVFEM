\documentclass[aps,preprint]{revtex4}
\usepackage{amssymb}
\usepackage{amsfonts}
\usepackage{amsmath}
\setcounter{MaxMatrixCols}{30}
\usepackage{graphicx}
\providecommand{\U}[1]{\protect\rule{.1in}{.1in}}
\begin{document}
\title{Simulation of Nucleation and Growth in Periodically Precipitating Systems on
Structureless Meshes}
\author{Andrew Abi Mansour}
\affiliation{Department of Chemistry, Indiana University, Bloomington, USA}
\author{Mazen Al-Ghoul}
\affiliation{Department of Chemistry and Program in Computational Science,
American University of Beirut, Beirut, Lebanon}

\begin{abstract}
A computational method is suggested for the simulation of Liesegang
patterns in 2D on structureless meshes. The model incorporates nucleation and
growth of solid particles into reaction-diffusion equations. Spatial
discretization is carried out with the control volume finite element method (CVFM),
and a suitable time integrator based on backward differentiation formulas (BDF) is
constructed to solve the resultant stiff ordinary differential equations.  In
order to solve the non-linear system of equations, the
BDF integrator is coupled to a fast and robust scheme based on operator
splitting and a line search Jacobian-free Newton-Krylov method. The numerical solution is
compared with specially designed experiments on Liesegang patterns in various geometries 
and is shown to be in good agreement.
\end{abstract}
\maketitle


\section{Introduction}

Chemical reactions coupled to diffusion often produce a spatiotemporal
phenomenon that leads to the formation of a localized pattern. One classical
example of pattern formation in chemistry is periodic precipitation or the
so-called Liesegang phenomenon \cite{Chemische,Crystals,CoOH,Electric}. It is
a special type of chemical pattern formation that results from periodic
precipitation in certain diffusion-driven chemical systems. The pattern is
generally considered to be the product of a complex combination of chemical
reaction(s), diffusion, nucleation, and growth. Models explaining the
Liesegang phenomenon can be generally divided into two groups. The first is
based on Ostwald's theory of successive supersaturation where it is assumed
that the appearance of precipitation bands is due to a repetitive appearance
of a supersaturation wave \cite{Ostwald}. Since the system is diffusion
controlled, the reaction zone becomes depleted and therefore the time required
for precipitation to occur again increases. This in effect leads to an
increasing distance of separation between the bands. Theories based on this
concept are said ton follow pre-nucleation mechanisms because emphasis is
given to the relation among supersaturation, nucleation rate, and growth
kinetics. The second group of models is based on post-nucleation mechanisms
where emphasis is given to the processes occurring after the nucleation of the
solid phase. In such theories, it is generally believed that the growth
mechanism is the result of the Lifshitz-Slyozov instability \cite{Lifshitz}:
the growth of larger particles at the expense of the neighboring smaller ones,
which could dissolve back. In this case the pattern is described as a
dissipative structure without any need for supersaturation. A model which
takes both mechanism into account was introduced in \cite{precipitation}, and
it was shown \cite{Muller-Poly} that the model could reproduce the Liesegang
pattern on square domains. In this paper, a numerical scheme is proposed in
order to extend the model applicability to structureless meshes. In specific,
it is shown that the Liesegang pattern can be reproduced on circular domains
and starting from initial conditions of various geometrical constraints. This paper is composed of the following sections: section II contains the theoretical model describing the Liesegang phenomenon. In section III we present the numerical algorithm to solve the resulting partial differential equations. In section IV we implement the numerical scheme and present the numerical solutions and then compare them to the experiment. Section V concludes the paper.

\section{Theoretical model}

Consider a closed system $\Omega\subset\mathbb{R}^{N}$ bounded by
$\partial\Omega$ at any time $t \in[t_{0},\infty)$ such that $N$ is the
spatial dimension number of the system. Let the functions $a(\mathbf{x},t)$
and $b(\mathbf{x},t)$ represent the concentrations of two reactants, (A) and
(B), respectively. The vector $\mathbf{x} \in\mathbb{R}^{N}$ denotes position
in real space. The system $\Omega$ is partitioned into two; an interior
subdomain is where reactant $(A)$ is homogeneously distributed with
concentration $a_{0}$ and the reactant $(B)$ is zero, and an exterior part
where $(A)$ is zero and $(B)$ is homogeneously distributed with concentration
$b_{0}$. At time $t>t_{0}$ the two reactants begin to diffuse and react
irreversibly according to the following bimolecular chemical reaction
\begin{equation}
A+B\overset{k_{r}}{\rightarrow}C.
\end{equation}
The reaction constant is $k_{r}$ such that the rate of the reaction $v_{r}$ is
approximated by the mean-field theory
\begin{equation}
v_{r}(\mathbf{x},t)=k_{r}a(\mathbf{x},t)b(\mathbf{x},t).
\end{equation}
Let $c(\mathbf{x},t)$ denote the concentration function of (C), a weakly
soluble electrolyte. Evidently the initial value of (C) is zero everywhere on
$\Omega$. Over the boundaries, the fluxes of all components must vanish. The
Neumann boundary conditions for the three components are
\begin{align}
\left.  \mathbf{\nabla}_{\mathbf{x}}a\cdot\mathbf{n}\right\vert _{\partial
\Omega} =\left.  \mathbf{\nabla}_{\mathbf{x}}b\cdot\mathbf{n}\right\vert
_{\partial\Omega} =\left.  \mathbf{\nabla}_{\mathbf{x}}c\cdot\mathbf{n}%
\right\vert _{\partial\Omega} =0,
\end{align}
where $\mathbf{n}$ is the normal vector to $\partial\Omega$ and pointing
outwards. The $i^{th}$ component of the gradient $\nabla_{\mathbf{x}}%
\in\mathbb{R}^{N}$ is $\partial_{x_{i}}~\forall~i\in\left[  1,N\right]  $.
Taking into account that (A), (B), and (C) undergo diffusion in the absence of
convection, the following mass balance equations can be written as
\begin{align}
\partial_{t}a &  =\mathbf{\nabla}_{\mathbf{x}}\cdot\Gamma_{1}\mathbf{\nabla
}_{\mathbf{x}}a-v_{r}, \displaybreak[0]\label{rda}\\
\partial_{t}b &  =\mathbf{\nabla}_{\mathbf{x}}\cdot\Gamma_{2}\mathbf{\nabla
}_{\mathbf{x}}b-v_{r},\label{rdb}\\
\partial_{t}c &  =\mathbf{\nabla}_{\mathbf{x}}\cdot\Gamma_{3}\mathbf{\nabla
}_{\mathbf{x}}c+v_{r}+\mathbf{\Sigma}.\label{rdc}%
\end{align}
The scalar function $\mathbf{\Sigma}$ contains terms that are related to the
solid phase dynamics. The set $\left\{ \Gamma_{i} \right\} _{i=1}^{3}$
contains the diffusion coefficients of $(A),(B),$ and $(C)$, respectively. The
precipitate is assumed to exist in only two forms: small-sized nuclei
(represented by the nuclei distribution function $p_{1}(\mathbf{x},t)$) which
can dissolve back to replenish $(C)$ or transit to large-sized particles
(represented by $p_{2}(\mathbf{x},t)$), which can grow at the expense of
$(C)$. It was shown in \cite{precipitation} that $\mathbf{\Sigma}$ can take
the following simple form
\begin{align}
\mathbf{\Sigma}  & = v_{d} - v_{n} + v_{g},
\end{align}
such that $v_{d}$ is the rate of nuclei dissolution, $v_{n}$ is the rate of
nucleation, and $v_{g}$ is the rate of growth of the large particles. It is
further assumed that the precipitate does not diffuse, which makes the
evolution equations of the solid phase free of any terms containing spatial
derivatives, and that in turn makes their numerical solution less expensive.
This is physically justified on the basis that diffusion in solids is much
slower than in liquids, and the nucleation and growth timescales are much less
than that of diffusion. This agrees with experimental observations that show
that the precipitate formed in Liesegang systems does not diffuse and distort the
pattern \cite{Muller2,Muller3}. On the other hand, the particular forms of the
nucleation and growth rate functions do not significantly alter the final
pattern observed \cite{Muller-Poly,Heureux}. Therefore, the simplest possible
shaping functions are chosen such that the equations remain mass-balanced. The
nucleation and growth rate functions are given by
\begin{align}
v_{n} &  = k_{n}\left(  c-c_{n}\right)  H\left( c-c_{n}\right)  ,\\
v_{t} &  = k_{t}\left(  c-c_{d}\right)  H\left( c-c_{d}\right)  p_{1} ,\\
v_{d} &  = k_{d}\left(  c_{d} - c\right)  H\left( c_{d} - c\right)  p_{2},\\
v_{g} &  = k_{g} \left(  c - c_{g}\right)  H\left( c - c_{g}\right)  p_{2}.
\end{align}
The nucleation rate of the smallest particles, $v_{n}$, is independent of
$p_{1} \left(  \mathbf{x},t\right)  $ because otherwise nucleation cannot
commence. Moreover, the nucleation rate should vanish when $c\left(
\mathbf{x},t\right)  $ is less than some critical constant, termed $c_{n}$,
which is why the rate functions have been multiplied by the Heaviside
unit-step function. Similarly, the rate of transition from small nuclei to
large particles and that of dissolution depend on the critical dissolution
constant, $c_{d}$. Finally, the growth rate $v_{g}$, is a function of the particle
distribution function, $p_{2}$ and the critical growth constant, $c_{g}$. The
differential equations describing the precipitate dynamics can be written as
\begin{align}
\partial_{t} p_{1}  &  =k_{n} \left(  c-c_{n} \right)  H\left(  c-c_{n}
\right)  - k_{n} \left(  c-c_{d} \right)  H\left(  c-c_{d} \right)
p_{1}\nonumber\\
&  - k_{d} \left(  c_{d}-c\right)  H\left(  c_{d}-c\right)  p_{1}
,\label{nuclei}\\
\partial_{t}p_{2}  &  =k_{n}\left(  c-c_{d}\right)  H\left(  c-c_{d}\right)
p_{1}+k_{g}\left(  c-c_{g}\right)  H\left(  c-c_{g}\right)  p_{2}
.\label{particles}
\end{align}
The critical constants $c_{g}$ and $c_{d}$ are characteristic values of the
system such that when $c\left(  \mathbf{x},t\right)$ exceeds $c_{g}$, the
growth rate is greater than zero, otherwise this rate vanishes. On the other
hand, when $c\left(  \mathbf{x},t\right)$ is less than $c_{d}$ the rate of dissolution is
positive, otherwise it goes to zero. It is assumed that once the particles
reach their maximum size, they do not dissolve back. They can grow or remain
static in time. In this case, the Lifshitz-Slyozov instability \cite{Lifshitz}
leads to a constraint on the nucleation critical constants: $c_{3} > c_{2}
\geq c_{1}$. This constraint shall be imposed whenever Liesegang bands are sought after.

\section{Numerical analysis}

\subsection{Spatial discretization}

Eqs. (\ref{rda}-\ref{rdc},\ref{nuclei}-\ref{particles}) are spatially
discretized using the control volume finite element method (CVFEM), a general
class of finite volume methods (FVM) that are widely employed for transport
equations. CVFEM was introduced by Baliga and Patankar in 1980 \cite{CVFEM1}.
The method was initially applied to the solution of a two-dimensional
convection diffusion equation. Application to the complete Navier-Stokes
equations followed later\cite{CVFEM2}. An attractive feature of this
method is that the discretized equations are mass conservative and the non-linear
source or sink terms can be naturally diagonalized. This allows greater
flexibility in dealing with systems of non-linear equations over structureless
meshes. The numerical scheme used in this work is based on treating nucleation
and growth sites at the centers of the control volumes so that the non-linear
terms form diagonal matrices that can be exploited to add numerical stability.
Eqs. (\ref{rda}-\ref{rdc},\ref{nuclei}-\ref{particles}) are solved over a
two-dimensional mesh partitioned into a finite number of triangles. The
numerical solution of these equations is sought on the domain of each control
volume (CV), which is defined to be a $2m$ sided polygon where $m$ is the
number of triangles that contribute to the formation of each CV. Such a
polygon should be centered around each node of the mesh so that the total
number of CVs is equal to the total number of nodes, which is equal to $N_{c}%
$. Furthermore, the numerical solution $\phi$ is defined as a linear
interpolation of its values at three nodes of each triangle
\begin{equation}
\phi(\zeta) = \sum_{i=1}^{3} \zeta_{i} \phi_{i},\label{interpolation}%
\end{equation}
where $\{ \zeta_{i} \} _{i=1}^{3}$ is the set of basis functions [add
reference here], taken to be linear functions of space. Furthermore, the union
of all CV polygons must form the total mesh. For each triangle, its
contributing node belongs to two of its three edges. Thus, the midpoints of
each of these two edges are connected to the centroid of the triangle, forming
two segments of the CV polygon. The procedure is repeated again for another
contributing triangle, and so on. The result is a polygon with $2m$ sides
(Fig. (\ref{fig:CV1})) and of volume $V$,
\begin{equation}
V=\frac{1}{3}h\times\sum_{k=1}^{m}A_{k},
\end{equation}
where $h$ is the depth and $A_{k}$ the area of triangle $k$. Eqs.
(\ref{rda}-\ref{rdc},\ref{nuclei}-\ref{particles}) are transformed into their
weak form by integrating them over the entire domain.
\begin{equation}
\int_{\Omega_{k}} \partial_{t} \phi dV = \int_{\Omega_{k}} \nabla\cdot
\Gamma\nabla\phi dV + \int_{\Omega_{k}} s dV,
\end{equation}
where $\phi$ represents the concentration function of any of the components,
and $s$ a non-linear term that accounts for the reaction, nucleation, or
growth. The transient and reaction or nucleation integrals are then
approximated by a first order Gauss quadrature, which is evaluated at the
centroid of the CV (see Fig. (\ref{fig:CV1}))
\begin{equation}
\int_{\Omega_{k}} \partial_{t} \phi dV = A_{k} d_{t} \phi_{c} \times h,
\end{equation}
where $\phi_{c}$ is the value of $\phi$ at the centroid of the CV. Similarly,
the volume integral arising from the non-linear terms can be evaluated with a
first order Gauss quadrature
\begin{equation}
\int_{\Omega_{k}} s dV = A_{k} s(\phi_{c}) \times h.
\end{equation}
On the other hand, by applying the divergence theorem, each diffusion volume
integral is transformed into a surface integral which is split into two line
integrals
\begin{equation}
\int\Gamma\mathbf{\nabla}_{\mathbf{\zeta}} \phi\cdot\mathbf{n}dS=h\times
\int_{f_{1}}\Gamma\mathbf{\nabla}_{\mathbf{\zeta}} \phi\cdot\mathbf{n}
d\ell+h\times\int_{f_{2}}\Gamma\mathbf{\nabla}_{\mathbf{\zeta}} \phi
\cdot\mathbf{n}d\ell.
\end{equation}
Each line integral is approximated with a first order Gauss
quadrature
\begin{align}
\left.  \Gamma\mathbf{\nabla}_{\mathbf{\zeta}} \phi\cdot\mathbf{n} _{f_{1}%
}\ell\right\vert _{f_{1}} &  =\frac{\Gamma}{2A^{(k)}}\left\{  \left[  \left(
\mathbf{\zeta}_{2}^{k_{2}}-\mathbf{\zeta}_{2}^{k_{3}}\right)  \left(
\phi_{k_{1}}-\phi_{k_{3}}\right)  -\left(  \mathbf{\zeta}_{2}^{k_{1}}-
\mathbf{\zeta}_{2}^{k_{3}}\right)  \times\right.  \right.
\displaybreak[0]\nonumber\\
&  \left.  \left(  \phi_{k_{2}}-\phi_{k_{3}}\right)  \right]  \delta
\mathbf{x}_{2}^{(1)} +\left[  \left(  \mathbf{\zeta}_{1}^{k_{2}}%
-\mathbf{\zeta}_{1}^{k_{3}}\right)  \left(  \phi_{k_{1}}-\phi_{k_{3}}\right)
\right.  \displaybreak[0]\nonumber\\
&  \left.  \left.  -\left(  \mathbf{\zeta}_{1}^{k_{1}}-\mathbf{\zeta}
_{1}^{k_{3}}\right)  \left(  \phi_{k_{2}}-\phi_{k_{3}}\right)  \right]
\delta\mathbf{x}_{1}^{(1)}\right\}  , \displaybreak[0]\label{Assemble1}%
\end{align}
\begin{align}
\left.  \Gamma\mathbf{\nabla}_{\mathbf{\zeta}} \phi\cdot\mathbf{n}_{f_{2}}%
\ell\right\vert _{f_{2}} &  =\frac{\Gamma}{2A^{(k)}}\left\{  \left[  \left(
\mathbf{\zeta}_{2}^{k_{2}}-\mathbf{\zeta}_{2}^{k_{3}}\right)  \left(
\phi_{k_{1}}-\phi_{k_{3}}\right)  -\left(  \mathbf{\zeta}_{2}^{k_{1}%
}-\mathbf{\zeta}_{2}^{k_{3}}\right)  \times\right.  \right.
\displaybreak[0]\nonumber\\
&  \left.  \left(  \phi_{k_{2}}-\phi_{k_{3}}\right)  \right] \delta
\mathbf{x}_{2}^{(2)} +\left[  \left(  \mathbf{\zeta}_{1}^{k_{2}}%
-\mathbf{\zeta}_{1}^{k_{3}}\right)  \left(  \phi_{k_{1}}-\phi_{k_{3}}\right)
\right.  \displaybreak[0]\nonumber\\
&  \left.  -\left(  \mathbf{\zeta}_{1}^{k_{1}}-\mathbf{\zeta}_{1}^{k_{3}%
}\right)  \left(  \phi_{k_{2}}-\phi_{k_{3}}\right)  \delta\mathbf{x}_{1}%
^{(2)}\right\}  . \displaybreak[0]\label{Assemble2}%
\end{align}
The projection vectors from Fig. (\ref{fig:CV2}) can be found\cite{FEM} to be
\begin{align}
\delta\mathbf{x}^{(1)}  & = \frac{\mathbf{\zeta}^{k_{2}}}{3} - \frac
{\mathbf{\zeta}^{k_{3}}}{6} - \frac{\mathbf{\zeta}^{k_{1}}}{6},\\
\delta\mathbf{x}^{(2)}  & = - \frac{\mathbf{\zeta}^{k_{3}}}{3} +
\frac{\mathbf{\zeta}^{k_{2}}}{6} + \frac{\mathbf{\zeta}^{k_{1}}}{6}.
\end{align}
The diffusion surface integral yields%
\begin{equation}
\int\Gamma\mathbf{\nabla}_{\mathbf{\zeta}} \phi\cdot\mathbf{n}dS\simeq
h\times\sum_{i \in\Omega_{k}} c_{k_{1},i} \phi_{i},
\end{equation}
where
\begin{align}
c_{k_{1},k_{1}} &  =\frac{\Gamma}{2A_{k}}\left[ \left(  \mathbf{\zeta}^{k_{3}
}-\mathbf{\zeta}^{k_{2} }\right)  \cdot\left(  \delta\mathbf{x}^{(1)} +
\delta\mathbf{x}^{(2)} \right)  \right]  , \displaybreak[0]\\
c_{k_{1},k_{2}} &  =\frac{\Gamma}{2A_{k}}\left[  \left(  \mathbf{\zeta}^{
k_{1} }-\mathbf{\zeta}^{k_{3} }\right)  \cdot\left(  \delta\mathbf{x}^{(1)} +
\delta\mathbf{x}^{(2)} \right)  \right]  , \displaybreak[0]\\
c_{k_{1},k_{3}} &  =\frac{\Gamma}{2A_{k}}\left[  \left(  \mathbf{\zeta}^{k_{2}
}-\mathbf{\zeta}^{k_{1}}\right)  \cdot\left(  \delta\mathbf{x}^{(1)} +
\delta\mathbf{x}^{(2)} \right)  \right] .
\end{align}
Each of the faces of the polygon has an outward normal vector that is
perpendicular to its surface line (see Fig. (\ref{fig:CV2})). Since each
triangle provides two faces for the control volume, two fluxes are defined
with each being parallel to its outward normal vector. The global diffusion
matrix $\mathbf{\Pi}$ is assembled by looping over each node and updating
three entries. The number of triangles in the region of support can be
computed from the data structure generated by the open source software
\textit{Triangle }\cite{Triangle}, which generates a set of 2D arrays
(triangular nodes and their coordinates) that define a given mesh. For control
volume finite element computations, it is necessary that a region of support for
each CV is determined. An additional data structure therefore includes a
two-dimensional array of length $N_{c}$ and a variable number of columns.

Let $\mathbf{\Phi}$ represent the discretized unknown functions at all nodes.
Then, the discretized (time-dependent) ordinary differential equations for
Eqs. (\ref{rda}-\ref{rdc},\ref{nuclei}-\ref{particles}) can be written as
\begin{equation}
d_{t}\mathbf{\phi=}-\mathbf{\Psi\phi+s,}\label{ODEs}%
\end{equation}
where the tensor $\mathbf{\Psi}$ contains the diffusion matrices and the
vector $\mathbf{s}$ contains all the non-linear terms.

\subsection{Markovian time integration}

A variable order and variable step-size memory-based implicit time integrator
is used to integrate in time Eqs. (\ref{ODEs}) using up to $n^{th}$ order
backward differentiation formulas (BDF) derived from a linear sequence
acceleration; by applying Richardson extrapolation to Taylor series expansions
of $\phi^{k}$ at the $k^{th}$ time step, one can show by mathematical induction that
\begin{equation}
d_{t}\mathbf{\phi}^{t_{k}} = \sum_{i=0}^{n}\lambda_{i}\mathbf{\phi}^{t _{k-i}
}+O\left(  \delta t ^{n}\right)  .\label{ODE}
\end{equation}
where $\delta t $ is related to the last $n$ time steps taken. The set
$\left\{  \lambda_{i}\right\} _{i=1}^{n}$ contains the BDF coefficients, the
values of which depend in a complicated manner on the previous and current
time steps taken, and the order $n$ of the BDF used. Classical BDF integrators
for stiff equations either take constant time steps \cite{Gear} and then
interpolate the previous solutions or take quasi-variable time steps keeping
the Jacobian intact for each Newton-like iteration. The first class of methods
do not have the same stability as variable-coefficient BDF integrators. The
so-called fixed-leading coefficient \cite{FLC1,FLC2} methods were therefore
developed later on such that they retained a similar stability to that of
variable-coefficient BDF. Such methods are efficient when the Jacobian does
not change with each iteration. On the other hand, variable-coefficient BDF
integrators did not perform well for the nucleation and growth equations
possibly because a quasi-Newton method is more prone to error. Such
integrators yielded unphysical solutions and therefore failed in simulating
the Liesegang pattern. This is because even a globally convergent Newton
method does not converge to the correct solution when the time discretization
method yields multiple solutions. This typically happens when the BDF
integrator takes a large time step [add a ref here], which means one way of
obtaining a unique numerical solution is by decreasing the time step. The BDF
integrator used in this work computes the BDF coefficients and a new $\delta
t$ for every single time step, and then it computes the reduced Jacobian for
every optimized Newton step, based on the Trust Region algorithm
\cite{Optimization}. an explicit time integration scheme (based on the simple
Euler method) is used to obtain an initial guess for Newton's method. This is
covered in more detail in the next subsection. For stiff equations, it is
suitable to vary the time step so that the minimum number of computations are
performed by the solver \cite{Shampine}. Controlling the time step is usually
based on finding an estimate of how accurate the numerical solution is to the
exact solution [add ref here]. Such a practical estimator is the local error
$e_{n}$, which is computed from the difference of the temporal derivatives of
$\mathbf{\phi}$ of orders $n$ and $n-1$. The new time step chosen by the
solver $\delta t _{k} $ can be estimated from the last time step $\delta t
^{\prime}$ attempted by the solver such that the local error $e_{n}$ is less
than some specified tolerance $\emph{tol}:$%
\begin{equation}
\delta t _{k}=\alpha\delta t ^{\prime}\left(  \frac{\emph{tol}}{e_{n}}\right)
^{\frac{1}{n}}.
\end{equation}
Since a failed step can be quite expensive, only a fraction of the complete
time step is taken. That is, $\alpha$ is set to be less than $1$. In all of
the simulations shown in this paper, $\alpha$ was taken to be $0.8$.


\subsection{Algebraic solution}

The reactants (A) and (B) form a reaction-diffusion system that is independent
of other components. Therefore, the differential equations for (A) and (B)
ought to be solved first, and the solution is used for the nucleation and
growth equations. Newton's method is the most commonly used iterative method
for finding an approximate solution of a system of non-linear algebraic
equations. For large computational problems, however, building the Jacobian
(or an approximation of it) can be quite expensive. Therefore, a fixed point
iteration (FPI) is used that allows decomposition and manipulation of the
non-linear terms in such a way that convergence is rapid for the (A)-(B)
reaction-diffusion system. Such a scheme falls under the realm of operator
splitting (OS) methods [add a ref here]. The linearized equations are
\begin{align}
\left[  \mathbf{\Upsilon_{A}^{t _{k}}}\right]  _{j}\left(  \delta\mathbf{a}^{t
_{k}} \right)  _{j+1}=-\left[  \delta\mathbf{\Upsilon_{A}^{t _{k}}}\right]
_{j}\left(  \delta\mathbf{a}^{t _{k}} \right)  _{j},\\
\left[  \mathbf{\Upsilon_{B}^{t _{k}}}\right]  _{j}\left(  \delta\mathbf{b}^{t
_{k}} \right)  _{j+1}=-\left[  \delta\mathbf{\Upsilon_{B}^{t _{k}}}\right]
_{j}\left(  \delta\mathbf{b}^{t _{k}} \right)  _{j},
\end{align}
where the notation $\delta\left[  \mathbf{v}\right]  _{j+1}$ is equal to
$\left[  \mathbf{v}\right]  _{j+1}-\left[  \mathbf{v}\right]  _{j}$ for a
tensor $\mathbf{v}$, the matrix $\mathbf{D}_{v}$ is a diagonal matrix equal to
$\mathbf{diag}\left(  \mathbf{v}\right)  $, the vectors $\mathbf{a}$ and
$\mathbf{b}$ contain the discretized concentrations of $(A)$ and $(B)$,
respectively, and
\begin{align}
\left[ \mathbf{\Upsilon_{A}^{t _{k}}}\right]  _{j} = \lambda_{0}
\mathbf{I+}{\Gamma}_{1}\mathbf{\Pi+}{k}_{r} \left[  \mathbf{D}_{b}^{t _{k}
}\right]  _{j},\label{RD_OPA}\\
\left[ \mathbf{\Upsilon_{B}^{t _{k}}}\right]  _{j} = \lambda_{0}
\mathbf{I+}{\Gamma}_{2}\mathbf{\Pi+}{k}_{r}\left[  \mathbf{D}_{a}^{t _{k}
}\right]  _{j}, \label{RD_OPB}
\end{align}
where $\mathbf{\Pi}$ is the diffusion tensor. The convergence rate of this
method is obviously linear. However, since convergence is quite rapid, the
method is computationally less expensive than the Newton-Raphson. On the other
hand, the nucleation, growth, and diffusion equations for (C), (N), and (P)
are solved using the Trust Region \cite{Optimization} algorithm, where a
modified Newton direction is taken, depending on how well the right hand-side
equations are minimized. The unconstrained Newton method leads to the
following linearized system
\begin{equation}
\left[  \mathbf{J}\right]  _{j}\mathbf{\times}\left(
\begin{array}
[c]{c}
\delta\mathbf{c}\\
\delta\mathbf{n}\\
\delta\mathbf{p}
\end{array}
\right)  _{j+1}=-\left(
\begin{array}
[c]{c}
\sum_{i=0}^{n}\lambda_{i}\mathbf{c}^{t _{k}-i} + {\Gamma}_{3}\mathbf{\Pi c}^{t
_{k}} - \mathbf{s}_{c}\\
\sum_{i=0}^{n}\lambda_{i}\mathbf{n}^{t _{k}-i} - \mathbf{s}_{n}\\
\sum_{i=0}^{n}\lambda_{i}\mathbf{p}^{t _{k}-i} - \mathbf{s}_{p}
\end{array}
\right)  _{j},
\end{equation}
where $\mathbf{s}_{c}$, $\mathbf{s}_{n}$, and $\mathbf{s}_{p}$ are vectors
that contain all the discretized non-linear terms for species $(C)$, $(N)$,
and $(P)$, respectively. The Jacobian $\mathbf{J}$ of these equations can be
written in the following form:
\begin{equation}
\mathbf{J}=\left(
\begin{array}
[c]{ccc}
\mathbf{J}_{11} & \mathbf{J}_{12} & \mathbf{J}_{13}\\
\mathbf{J}_{21} & \mathbf{J}_{22} & \mathbf{0}\\
\mathbf{J}_{31} & \mathbf{J}_{32} & \mathbf{J}_{33}
\end{array}
\right)  ,
\end{equation}
where
\begin{align}
\mathbf{J}_{11}  & =\lambda_{0}\mathbf{I}+{\Gamma}_{3}\mathbf{\Pi+ }{k}%
_{n}^{(1)}\mathbf{\theta}^{(1)}+{k}_{g}\mathbf{\theta}^{(3)}\mathbf{D}_{p}%
^{k}+{k}_{d}\left(  \mathbf{I-\theta} ^{(2)}\right)  \mathbf{D}_{n}^{k},
\displaybreak[0]\\
\mathbf{J}_{12}  & =-{k}_{d}\left(  {c}_{d} \mathbf{I} -\mathbf{D}
_{c}\right)  \left(  \mathbf{I} -\mathbf{\theta}^{(2)}\right)  ,\\
\mathbf{J}_{13}  & ={k}_{g}\left(  \mathbf{D}_{c}-{c} _{g} \mathbf{I} \right)
\mathbf{\theta}^{(3)},\\
\mathbf{J}_{21}  & =-{k}_{n}^{(1)}\mathbf{\theta}^{(1)}+{k} _{d}\left(
\mathbf{I} -\mathbf{\theta}^{(2)}\right)  \mathbf{D}_{n}+{k} _{n}%
^{(2)}\mathbf{\theta}^{(2)}\mathbf{D}_{n},\\
\mathbf{J}_{22}  & =\lambda_{0}\mathbf{I+}{k}_{d}\left(  {c} _{d} \mathbf{I}
-\mathbf{D}_{c}\right)  \left(  \mathbf{I} -\mathbf{\theta}^{(2)}\right)  +{
k}_{n}^{(2)}\mathbf{\theta}^{(2)}\left(  \mathbf{D}_{c}-{c} _{d} \mathbf{I}
\right)  \mathbf{,}\\
\mathbf{J}_{31}  & =-{k}_{n}^{(2)}\mathbf{\theta}^{(2)}\mathbf{D} _{n}-{k}%
_{g}\mathbf{\theta}^{(3)}\mathbf{D}_{p},\\
\mathbf{J}_{32}  & =-{k}_{n}^{(2)}\left(  \mathbf{D}_{c}-{c} _{d} \mathbf{I}
\right)  \mathbf{\theta}^{(2)},\\
\mathbf{J}_{33}  & =\lambda_{0}\mathbf{I}-{k}_{g}\left(  \mathbf{D} _{c}%
-{c}_{g} \mathbf{I} \right)  \mathbf{\theta}^{(3)},
\end{align}
where $\mathbf{\theta^{(i)}} \forall i \in[1,3]$ is a diagonal matrix of
entries $\theta^{(i)}_{j} = H(\mathbf{c}_{j}-c_{i})$. Since the differential
equations describing the temporal evolution of the nuclei and the precipitate
in Eqs. (\ref{nuclei},\ref{particles}) are free of spatial derivatives
($\mathbf{J}_{22}$ and $\mathbf{J}_{33}$ are diagonal matrices), the Jacobian
can be reduced in size from $3N_{c}\times3N_{c}$ to $N_{c}\times N_{c}:$
\begin{equation}
\left[  \mathbf{J}_{c}\right]  _{j}\left[  \delta\mathbf{c}\right]
_{j+1}\mathbf{=-}\left[  \mathbf{r}_{c}\right]  _{j},
\end{equation}
where
\begin{equation}
\mathbf{J}_{c}=\mathbf{J}_{11}-\mathbf{J}_{12}\mathbf{J}_{21}\mathbf{J}%
_{22}^{-1}-\mathbf{J}_{13}\mathbf{J}_{31}\mathbf{J}_{33}^{-1}+\mathbf{J}%
_{13}\mathbf{J}_{32}\mathbf{J}_{21}\mathbf{J}_{22}^{-1}\mathbf{J}_{33}^{-1},
\end{equation}%
\begin{align}
\mathbf{r}_{c}  & =\sum_{i=0}^{n}\lambda_{i}\mathbf{c}^{t _{k}-i}+{\Gamma}%
_{3}\mathbf{\Pi c-s}_{c}-\mathbf{J}_{12}\mathbf{J}_{22}^{-1}\mathbf{r}%
_{n}-\mathbf{J}_{13}\mathbf{J}_{33}^{-1}\mathbf{r}_{p}\nonumber\\
& +\mathbf{J}_{13}\mathbf{J}_{32}\mathbf{J}_{22}^{-1}\mathbf{J}_{33}%
^{-1}\mathbf{r}_{n},
\end{align}
and
\begin{align}
\mathbf{r_{n}}\ = \sum_{i=0}^{n}\lambda_{i}\mathbf{n}^{t _{k}-i} - s_{n},
\end{align}
\begin{align}
\mathbf{r_{p}}\ = \sum_{i=0}^{n}\lambda_{i}\mathbf{p}^{t _{k}-i} - s_{p}.
\end{align}
From $\left[  \delta\mathbf{c}\right]  _{j+1}$ one can easily compute $\left[
\delta\mathbf{n}\right]  _{j}$ and $\left[  \delta\mathbf{p}\right]  _{j}$ as
such:%
\begin{align}
\left[  \delta\mathbf{n}\right]  _{j}  & = - \left[  \mathbf{J}_{22}%
^{-1}\right]  _{j}\times\left(  \left[  \mathbf{r}_{n}\right]  _{j}+\left[
\mathbf{J}_{21}\right]  _{j}\left[  \delta\mathbf{c}\right]  _{j}\right)  ,\\
\left[  \delta\mathbf{p}\right]  _{j}  & = - \left[  \mathbf{J}_{33}%
^{-1}\right]  _{j}\times\left(  \left[  \mathbf{r}_{p}\right]  _{j}+\left[
\mathbf{J}_{32}\right]  _{j}\left[  \delta\mathbf{n}\right]  _{j}+\left[
\mathbf{J}_{31}\right]  _{j}\left[  \delta\mathbf{c}\right]  _{j}\right)  ,
\end{align}
where the notation $[\mathbf{v}]_{j}$ denotes that the tensor $\mathbf{v}$ in
brackets is being evaluated at the $j^{th}$ iteration. It is worth noting that
the reduced Jacobian is not only smaller in size but also symmetric. All of
the discretized matrices that result from Eqs. (\ref{rda}-\ref{rdc}%
,\ref{nuclei}-\ref{particles}) are sparse. However, only the assembled
diffusion matrix is stored. We have used CSparse \cite{sparse} package for
sparse storage (in compressed column format) and for linear algebra floating
point operations. For each iteration, three linearized systems had to be
solved. Since all three operators are symmetric positive definite (spd), the
preconditioned conjugate gradient (PCCG) method \cite{Saad} is used. Since the
most expensive BLAS in iterative solvers is a matrix-vector product, the
matrices need not be constructed or stored. Instead, the matrix-vector product
is computed by updating the diagonal entry of each column of the assembled
diffusion matrix by adding the source or sink term and then the dot product is
computed. Such an implementation falls under the class of Matrix-free
Newton-Krylov methods \cite{NewtonKrylov}. While these methods are efficient
because the operator is not stored, the construction of a suitable
preconditioner becomes a challenge and is the subject of ongoing research.
For diagonally dominant matrices, such as those that arise from
reaction-diffusion equations, the most inexpensive and effective
preconditioner is the Jacobi which can be analytically computed and stored in
vector form without the need to construct the reaction diffusion operators.
However, when the ratio of the reaction (and nucleation or growth) timescale
to that of diffusion is close to unity, the Jacobi preconditioner increases
the condition number of the operator, which in this case does not need to be
preconditioned. This is because the matrices are well-conditioned and
convergence is rapid with the conjugate gradient (CG) method. The solver
selects PCCG or CG based on the ratio $\left(  {\max\{k_{r},k_{g},k^{(1)}%
_{n},k_{n}^{(2)},k_{d}\}} \right) / \left( {\min\, \{\Gamma_{i}\}_{i=1}^{3} }
\right) $. Convergence of this scheme is rapid and is $O(n)$ in memory storage
and floating point operations.

\section{Results and discussion}

\subsection{Numerical solutions}

The numerical algorithm is implemented and tested in different domains of
initially separated reactants (A and B) and by choosing suitable parameters
that give rise to Liesegang bands. The inner electrolyte (A), which assumes
the higher concentration, occupies inner domains of specific shapes and/or
configuration, and the outer electrolyte (B), which has the relatively lower
concentration, occupies the remaining domains. The shape of the overall domain
in this work is either circular or squared and we choose various shapes and
configurations for the inner domains. The choice of the geometrical shapes and
configurations are chosen in such a way that can be directly tested
experimentally, but they can assume any shape or configuration. The spatial
distribution function for the precipitate (P) is then plotted at a given time
$\tau$. In the case of a circular inner domain as shown in Fig. (\ref{fig:CV5}), we can see that the system develops into circular Liesegang rings with
increasing spacing as expected in such systems\cite{so79470}. This spacing
follows a so-called spacing law \cite{chemistryinmotion} and is also verified in this model but not
shown. Moreover, the time of appearance of the rings also follow a time law\cite{chemistryinmotion}
characteristic of Liesegang rings whereby the time it takes to form the
$n^{th}$ increases as a function of $n$ and is also not shown here. In the
case of Fig. (\ref{fig:CV6}), the reactant is initially distributed into four
equal circular containers. The resulting pattern is complex especially in the
region of intersection of the precipitates. In Fig. (\ref{fig:CV7}), we start
with a square domain and the reactant is initially distributed over nine
circular domains of equal size. We obtain a precipitation pattern that is
symmetric and complex with increasing details at the intersections of the precipitates.

In Fig. (\ref{fig:CV8}), we start with an inner domain with a
square geometry over a circular domain. In this case, the rings that form tend to round the 
edges cause by the inner square geometry. This feature is due to diffusion and will be reproduced
experimentally in the next section.  

In Fig. (\ref{fig:CV8}), we again change the inner domain from a square to a
triangle. The resulting precipitation pattern is a modulated triangle with round edges (similar to the 
square situation discussed previously), which
lead to a break-up due to strain imposed by the triangular geometry. The
breakup persists even after jiggling the mesh or by refining it. This feature is always encountered in 
experiment in the case of geometries with sharp edges and might give rise to dislocations as we will see in the coming section.

\subsection{Comparison with experiment}

To compare the simulation results with experiment, we pick the cobalt
hydroxide/ammonia Liesegang system\cite{alghoulcrys}. A gel (gelatin)
containing a cobalt chloride (source of Co$^{2+}$) displays the green/blue $\alpha-$Co(OH)$_{2}$
($\mathbf{K}_{\mathbf{sp}}=3.00\times10^{-16})$ precipitation Liesegang rings
upon diffusion of ammonia solution (called inner electrolyte) which consists of
ammonia (NH$_{3})$, the hydroxide ions (OH$^{-}$) and the ammonium ions
(NH$_{4}^{+}$). $\alpha-$Co(OH)$_{2}$ is one of two polymorphs of Cobalt
hydroxide\cite{polymorph}. Later, the Co(OH)$_{2}$ rings near the interface
redissolve, forming the orange/yellow Co(NH$_{3}$)$_{n}^{2+}$ complex (for
$n=6$, $K_{f}=5.00\times10^{4})$ due to a trailing wave of excess NH$_{3}$
\cite{lloyd}. However, in this paper, we only focus on ring formation, which
dominates due to a proper choice of experimental parameters, and ignore the
dissolution process. Consequently, the effective reaction scheme representing
the precipitation scenario in the presence of \ ammonia is:
\begin{equation}
\text{Co}^{+2}(aq)+2\text{OH}^{-}(aq)\rightarrow\alpha-\text{Co}
(\text{OH})_{2}(s)\rightarrow\text{~Liesegang~rings}.
\end{equation}


The preparation of the system consists of the following experimental
procedures: The required masses of CoCl$_{2}$.6H$_{2}$O (Fluka) and Gelatin
(5\%) (Difco) were weighed to the nearest 0.0001g. These masses are then
transferred to a beaker containing 5.00 mL of doubly distilled water. After
dissolution of the solid, 0.2500 grams of gelatin powder are added to make a
5\% gelatin solution. The solution is then heated with continuous stirring
until all the gelatin dissolves. The resulting pink Co(II)$/$gel mixture is
immediately transferred into a specially designed circular reactor which
consists of a plexiglass petridish-like container and in its center a
reservoir tube (or multi reservoirs) of a defined cross section is fitted. The
system is then covered with parafilm paper and allowed to stand for about 24
hours in a thermostatic chamber at $18{{}^{\circ}}$C. On the next day, the gel
is carefully removed from the bottom of the pouring reservoir tube then
$13.37$ M of ammonia solution is added to the reservoir. As the outer
electrolyte solution is being added over the interface, a homogeneous greenish
blue precipitate starts to form. Photos of the evolution of the precipitate
system are taken with a computer-interfaced digital Canon D450 camera at a
specified time interval. In Fig. (\ref{fig:experiment}A-D)), we show the
resulting pattern as we vary the shape of the cross section of the inner
reservoir (Fig. (\ref{fig:experiment}A-C)) or if we add more circular
reservoirs (Fig. (\ref{fig:experiment}D)). It is clear that different patterns
emerge as a consequence and whose shape is predicted by the simulation. This
confirms the ability to use the model and simulation as tools for design of
precipitation patterns for any geometry. 
Dislocations are also encountered in the case of inner geometries with edges are can be clearly see in 
Fig. \ref{fig:experiment}(B,C) for triangular and square geometries.


\section{Conclusion}

We develop a model and numerical algorithm for the simulation of Liesegang
patterns on structureless meshes. The method is based on the control volume
finite element method for spatial discretization and a time integrator based
on backward differentiation formulas. The resulting non-linear system of
equations is solved by using operator splitting and a constrained line
search Jacobian-free Newton-Krylov method. The numerical results are shown to
be in good agreement with the experiment for various geometrical
configurations. Therefore, this method may be used for the prediction and
design of complex precipitation patterns.

\acknowledgements{
The author would like to acknowledge the support of grants from the Lebanese
Council for Scientific Research (LCNSR) and from the University Research
Board, American University of Beirut.}




\bibliographystyle{plain}
\bibliography{CVFEM}

\newpage
\begin{figure}[htb]
\begin{center}
\includegraphics[scale=0.35]{figs/fig2.ps}
\end{center}
\caption{A Hexadecagonal control volume in grey (its centroid is in blue) is
generated from eight neighboring triangles that form the region of support.}
\label{fig:CV1}
\end{figure}

\begin{figure}[htb]
\begin{center}
\includegraphics[scale=0.35]{figs/fig3.ps}
\end{center}
\caption{A contributing triangle to the control volume (centered at node
$k_{1}$) through its two edges, $k_{1}$- $k_{2}$ and $k_{1}$-$k_{3}$. The
normal vectors to the faces are projected (in gray) along the $\mathbf{x}_{1}$
and $\mathbf{x}_{2}$ axes.}
\label{fig:CV2}
\end{figure}

\begin{figure}[htb]
\begin{center}
\includegraphics[scale=0.5]{figs/fig4.ps}
\end{center}
\caption{A plot of the time step $\delta t$ versus the number of time steps.
In solving Eqs. (\ref{ODEs}), the BDF integrator takes small timesteps during
the formation of a ring in order to resolve the non-linear nucleation and
growth terms. Between the formation of one ring and another, the system is
diffusion controlled, and the solver therefore takes much larger time steps.
The parameters used here are the same as those in Fig. (\ref{fig:CV5}) for the
Liesegang rings.}
\label{fig:BDF}
\end{figure} 

\begin{figure}[htb]
\begin{center}
\includegraphics[scale = 0.5]{figs/fig5.ps}
\end{center}
\caption{A mesh plot shows the precipitate (P) spatial distribution at $t
=800$ in the form of continuous Liesegang rings in a bounded system with two
initially separated reactants; (A) occupied an inner circle of radius
$\sqrt{200}$ while (B) filled the remaining part of the larger circle.
Parameters used: ${\Gamma} _{1}={\Gamma}_{2}=1$, ${\Gamma}_{3}=0.5$,
$a_{0}=500$, $b_{0}=3$, $k_{r}=10^{-3}$, $k_{n}^{(1)} = 2.0$, $k_{n}^{(2)} =
2.0$, $k_{d} = 1.0$, $k_{g} = 1.0$, $c_{1} = 1$, $c_{2} = 1.25$, $c_{3} =
1.5$. Mesh characteristics: $501,873$ vertices, $1,001,395$ triangles, and of
maximum area $= 0.05$. The precipitate global size distribution function has
been rescaled by a factor of 0.5.}
\label{fig:CV4}
\end{figure}

\begin{figure}[htb]
\begin{center}
\includegraphics[scale = 0.5]{figs/fig6.ps}
\end{center}
\caption{A mesh plot shows the precipitate (P) spatial distribution at $t
=300$ in the form of four Liesegang rings that intersect at the center of the
circular mesh. The reactant $(A)$ was initially distributed throughout four
separated circles of radii $\sqrt{150}$ on the same mesh and with the same
parameters used in Fig. (\ref{fig:CV5}). The precipitate global size
distribution function has been rescaled by a factor of 0.75.}
\label{fig:CV5}
\end{figure}

\begin{figure}[htb]
\begin{center}
\includegraphics[scale = 0.5]{figs/fig7.ps}
\end{center}
\caption{A mesh plot shows the precipitate (P) spatial distribution at $t
=200$ in the form of nine Liesegang rings. The reactant $(A)$ was initially
distributed throughout nine separated circles of radii $\sqrt{100}$ on a
square mesh and with the same parameters used in Fig. (\ref{fig:CV5}). Mesh
characteristics: $642035$ vertices, $1281543$ triangles, and of maximum area
$= 0.05$.}
\label{fig:CV6}
\end{figure}

\begin{figure}[htb]
\begin{center}
\includegraphics[scale = 0.5]{figs/fig8.ps}
\end{center}
\caption{A mesh plot shows the precipitate (P) spatial distribution at $t=800$
in the form of four Liesegang squares that are eventually (as the front
approaches the boundary) perturbed into ellipses due to the circular geometry
of the mesh. The reactant $(A)$ was initially distributed throughout a square
of length $\sqrt{200\pi}$ on the same mesh and with the same parameters used
in Fig. (\ref{fig:CV5}). The precipitate global size distribution function has
been rescaled by a factor of 0.25.}
\label{fig:CV7}
\end{figure}

\begin{figure}[htb]
\begin{center}
\includegraphics[scale = 0.5]{figs/fig9.ps}
\end{center}
\caption{A mesh plot shows the precipitate (P) spatial distribution at $t=850$
in the form of four Liesegang triangles. The reactant $(A)$ was initially
distributed throughout an equilateral triangle of length $10\times\sqrt{\pi}$
on the same circular mesh used in Fig. (\ref{fig:CV5}). Parameters used:
${\Gamma}_{1}={\Gamma}_{2}=1$, ${\Gamma}_{3}=1.0$, $a_{0}=500$, $b_{0}=3$,
$k_{r}=10^{-3}$, $c_{1}=1$, $c_{2}=1.25$, $c_{3}=1.5$, $k_{n}^{(1)}=2.0$,
$k_{n}^{(2)}=2.0$, $k_{d}=1.0$, $k_{g}=0.5$. The precipitate global size
distribution function has been rescaled by a factor of 0.2.}
\label{fig:CV8}
\end{figure}


\begin{figure}[htb]
\centering
\includegraphics[scale=0.6]{figs/fig10.ps}
\caption{Different precipitation patterns of cobalt hydroxide are obtained
with different cross sections of the inner reservoir. The geometry of the
setup for each pattern is sketched in the upper corner of each panel. The time
at which these patterns are photographed is about 24 hours. (A) Perfect
precipitation rings are obtained with a circular cross section. (B) Elliptic
patterns are obtained with a triangular cross section. (C) Modulated rings are
obtained with a square cross section. (D) A more complex pattern is obtained
by using 4 circular reservoirs of similar (but not identical) diameters.}
\label{fig:experiment}
\end{figure}

\end{document}
